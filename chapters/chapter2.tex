\chapter{Ecuaciones diferenciales lineales de orden superior}

\section{Valores iniciales y valores fronteras}
Let me clarify the difference between initial values and boundary values in differential equations.

Initial values specify the value of the function (and possibly its derivatives) at a single point, typically at the beginning of the interval you're considering (often at $t=0$). For example, in a problem like:

$dy/dt = 2y, y(0) = 1$

Here, $y(0) = 1$ is the initial value - it tells us where the solution starts.

Boundary values, on the other hand, specify values of the function at multiple points, typically at the endpoints of an interval. For example:

$d^2y/dx^2 + y = 0, y(0) = 0, y(\pi) = 1$

Here, we have two boundary values: one at $x=0$ and another at $x=π$. They constrain the solution at both ends of the interval [0,π].

The key differences are:
1. Initial value problems (IVPs) only specify conditions at one point, while boundary value problems (BVPs) specify conditions at multiple points
2. IVPs typically model time-evolution problems where you care about how something develops from a starting state
3. BVPs often arise in equilibrium or steady-state problems where you care about satisfying conditions at different locations

Would you like me to show you some more examples to illustrate these differences?

\section{ecuaciones caracteristicas}
Only can solve homogenous linear diferencial equations

\section{Solving linear homogeneous DE with constant coefficients}

Given a linear homogeneous differential equation with constant coefficients and its
characteristic equation, based on the latter we have the following cases:

\begin{itemize}
\item If $\lambda$ is a simple real root
$\Rightarrow e^{\lambda x}$ is a solution of the homogeneous DE.

\item If $\lambda$ is a real root with multiplicity $m > 1 \Rightarrow \{e^{\lambda x}, xe^{\lambda x}, x^2e^{\lambda x}, ..., x^{m-1}e^{\lambda x}\}$ is a set of solutions of the homogeneous DE associated with $\lambda$.

\item If $\lambda = \alpha + i\beta$ is a simple root $\Rightarrow \{e^{\alpha x}\cos(\beta x), e^{\alpha x}\sin(\beta x)\}$ is a solution set of the homogeneous DE associated with $\lambda$.
\end{itemize}



\section{order reduction method}
Solve linear ED which are homogenous and non-honogenous 


\section{Metodo coeficientes indeterminados}


\section{Metodo de anuladores}

\begin{table}[h]
\centering
\begin{tabular}{|c|c|}
\hline
$f(x)$ & Anulador \\
\hline
$P_n(x) = a_nx^n + a_{n-1}x^{n-1} + ... + a_1x + a_0$ & $D^{n+1}$ \\
\hline
$Ae^{\alpha x}$ & $D - \alpha$ \\
\hline
$A\cos(\omega x) + B\sin(\omega x)$ & $D^2 + \omega^2$ \\
\hline
$P_n(x)e^{\alpha x}$ & $(D - \alpha)^{n+1}$ \\
\hline
$P_n(x)\cos(\omega x) + Q_m(x)\sin(\omega x)$ & $(D^2 + \omega^2)^{N+1}, N = \max\{m, n\}$ \\
\hline
$e^{\alpha x}[A\cos(\omega x) + B\sin(\omega x)]$ & $(D - \alpha)^2 + \omega^2$ \\
\hline
$e^{\alpha x}[P_n(x)\cos(\omega x) + Q_m(x)\sin(\omega x)]$ & $[(D - \alpha)^2 + \omega^2]^{N+1}, N = \max\{m, n\}$ \\
\hline
\end{tabular}
\caption{Differential Operators and Their Annihilators}
\label{table:diff_operators}
\end{table}


\section{Metodo variacion de parametros}
Make use of the wronskian
