\chapter{Annexe 1 - Formulaire}

\section{Trigonometry formula}

\subsection{Derivative \& antiderivative rules}

\begin{table}[h]
\centering
\begin{tabular}{|c|c|}
\hline
\textbf{Derivative Rule} & \textbf{Antiderivative Rule} \\
\hline
$\dfrac{d}{dx}\sin x = \cos x$ & $\int \cos x \, dx = \sin x + C$ \\
\hline
$\dfrac{d}{dx}\cos x = -\sin x$ & $\int \sin x \, dx = -\cos x + C$ \\
\hline
$\dfrac{d}{dx}\tan x = \sec^2 x$ & $\int \sec^2 x \, dx = \tan x + C$ \\
\hline
$\dfrac{d}{dx}\cot x = -\csc^2 x$ & $\int \csc^2 x \, dx = -\cot x + C$ \\
\hline
$\dfrac{d}{dx}\sec x = \sec x\tan x$ & $\int \sec x\tan x \, dx = \sec x + C$ \\
\hline
$\dfrac{d}{dx}\csc x = -\csc x\cot x$ & $\int \csc x\cot x \, dx = -\csc x + C$ \\
\hline
$\dfrac{d}{dx}\arcsin x = \dfrac{1}{\sqrt{1-x^2}}$ & $\int \dfrac{1}{\sqrt{1-x^2}} \, dx = \arcsin x + C$ \\
\hline
$\dfrac{d}{dx}\arccos x = -\dfrac{1}{\sqrt{1-x^2}}$ & $\int -\dfrac{1}{\sqrt{1-x^2}} \, dx = \arccos x + C$ \\
\hline
$\dfrac{d}{dx}\arctan x = \dfrac{1}{1+x^2}$ & $\int \dfrac{1}{1+x^2} \, dx = \arctan x + C$ \\

\hline
\end{tabular}
\caption{Trigonometric Derivative and Antiderivative Rules}
\label{tab:trig-rules}
\end{table}


\subsection{Double angle and half angle formula}
\begin{table}[h!]
\centering
\renewcommand{\arraystretch}{1.5}
\begin{tabular}{|c|c|}
\hline
\textbf{Double Angle Formulas} & \textbf{Half Angle Formulas} \\ \hline
$\sin 2\theta = 2\sin\theta \cos\theta$ & $\sin^2\theta = 1 - \cos 2\theta$ \\ 
$\cos 2\theta = \cos^2\theta - \sin^2\theta$ & $\sin \frac{\theta}{2} = \pm \sqrt{\frac{1 - \cos\theta}{2}}$ \\
$= 2\cos^2\theta - 1$ & $\cos^2\theta = \frac{1 + \cos 2\theta}{2}$ \\ 
$= 1 - 2\sin^2\theta$ & $\cos \frac{\theta}{2} = \pm \sqrt{\frac{1 + \cos\theta}{2}}$ \\
$\tan 2\theta = \frac{2\tan\theta}{1 - \tan^2\theta}$ & $\tan \frac{\theta}{2} = \pm \sqrt{\frac{1 - \cos\theta}{1 + \cos\theta}}$ \\ \hline
\end{tabular}
\caption{Double and Half Angle Formulas}
\end{table}


\section{Number Sets in Mathematics}

In mathematics, we define several fundamental sets of numbers, each represented by a special bold uppercase letter:

\begin{itemize}
    \item $\mathbb{N}$ (Natural Numbers): The set of positive integers 
    \[
    \mathbb{N} = \{1, 2, 3, 4, \ldots\}
    \]
    Sometimes defined to include 0: $\mathbb{N}_0 = \{0, 1, 2, 3, \ldots\}$

    \item $\mathbb{Z}$ (Integers): All whole numbers, positive and negative 
    \[
    \mathbb{Z} = \{\ldots, -3, -2, -1, 0, 1, 2, 3, \ldots\}
    \]

    \item $\mathbb{Q}$ (Rational Numbers): Numbers that can be expressed as a fraction of two integers
    \[
    \mathbb{Q} = \{\frac{p}{q} : p, q \in \mathbb{Z}, q \neq 0\}
    \]

    \item $\mathbb{R}$ (Real Numbers): All rational and irrational numbers on the number line
    \[
    \mathbb{R} \text{ includes all points on a continuous number line}
    \]

    \item $\mathbb{C}$ (Complex Numbers): Numbers with real and imaginary parts
    \[
    \mathbb{C} = \{a + bi : a, b \in \mathbb{R}, i^2 = -1\}
    \]
\end{itemize}

These sets form a hierarchy of increasing complexity and inclusivity:
\[
\mathbb{N} \subset \mathbb{Z} \subset \mathbb{Q} \subset \mathbb{R} \subset \mathbb{C}
\]