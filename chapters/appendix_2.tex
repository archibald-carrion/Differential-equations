\chapter{Appendix 2 - Integration methods}

\section*{Integration by Partial Fractions}

Integration by partial fractions is a technique used to integrate rational functions, which are ratios of polynomials. The basic idea is to decompose a complex rational function into simpler fractions that are easier to integrate.

\subsection*{Steps for Integration by Partial Fractions}

\begin{enumerate}
    \item \textbf{Ensure Proper Fraction:} Make sure the degree of the numerator is less than the degree of the denominator. If not, perform polynomial long division first.
    
    \item \textbf{Factor the Denominator:} Factor the denominator of the rational function completely into linear and/or irreducible quadratic factors.
    
    \item \textbf{Set Up Partial Fractions:} Express the fraction as a sum of partial fractions. The form depends on the factors:
    \begin{itemize}
        \item For each linear factor $(ax + b)$, use a term $\frac{A}{ax+b}$
        
        \item For each repeated linear factor $(ax + b)^n$, use terms 
        $\frac{A_1}{ax+b} + \frac{A_2}{(ax+b)^2} + \cdots + \frac{A_n}{(ax+b)^n}$
        
        \item For each irreducible quadratic factor $(ax^2 + bx + c)$, use a term $\frac{Ax+B}{ax^2+bx+c}$
    \end{itemize}
    
    \item \textbf{Solve for Coefficients:} Multiply through by the common denominator to clear fractions, then equate the coefficients of corresponding powers of $x$ from both sides of the equation to solve for the unknown coefficients.
    
    \item \textbf{Integrate Each Term:} Once the original fraction is expressed as a sum of simpler fractions, integrate each term separately.
\end{enumerate}