\chapter{Solving DE using power series}

\section{Steps to Solve Differential Equations Using Power Series}

\subsection{1. Assume a Power Series Solution}
Assume that the solution $y(x)$ can be expressed as a power series centered at $x = x_0$:

\[y(x) = \sum_{n=0}^{\infty} a_n(x-x_0)^n\]

Here, $a_n$ are coefficients to be determined, and $x_0$ is typically the point about which the series is centered (often $x_0 = 0$).

\subsection{2. Differentiate the Power Series}
If the differential equation involves derivatives of $y(x)$, differentiate the power series term by term:

\[y'(x) = \sum_{n=1}^{\infty} na_n(x-x_0)^{n-1}\]

\[y''(x) = \sum_{n=2}^{\infty} n(n-1)a_n(x-x_0)^{n-2}\]

Continue this process for higher-order derivatives as needed.

\subsection{3. Substitute into the Differential Equation}
Substitute the power series expressions for $y(x)$, $y'(x)$, etc., into the differential equation.


\section{Classification of Points in Second-Order Linear Differential Equations}

Consider the general form of a linear second-order differential equation:
\[P(x)y'' + Q(x)y' + R(x)y = 0\]
where $P(x)$, $Q(x)$, and $R(x)$ are functions of $x$.

\subsection{Ordinary Points}
A point $x = x_0$ is called an \textbf{ordinary point} of the differential equation if:
\[P(x_0) \neq 0\]
At an ordinary point:
\begin{itemize}
    \item The functions $p(x) = \frac{Q(x)}{P(x)}$ and $q(x) = \frac{R(x)}{P(x)}$ are both analytic at $x_0$
    \item The equation can be written in standard form: $y'' + p(x)y' + q(x)y = 0$
    \item Solutions can be found using regular power series: $y = \sum_{n=0}^{\infty} a_n(x-x_0)^n$
\end{itemize}

\subsection{Singular Points}
A point $x = x_0$ is called a \textbf{singular point} if:
\[P(x_0) = 0\]

Singular points are further classified into two types:

\subsubsection{Regular Singular Points}
A singular point $x = x_0$ is \textbf{regular} if:
\begin{itemize}
    \item $(x-x_0)p(x)$ is analytic at $x = x_0$
    \item $(x-x_0)^2q(x)$ is analytic at $x = x_0$
\end{itemize}

This means the limits exist:
\[\lim_{x \to x_0} (x-x_0)p(x) \text{ and } \lim_{x \to x_0} (x-x_0)^2q(x)\]

Solutions near regular singular points can be found using the Frobenius method:
\[y = (x-x_0)^r\sum_{n=0}^{\infty} a_n(x-x_0)^n\]

\subsubsection{Irregular Singular Points}
A singular point $x = x_0$ is \textbf{irregular} if it is not regular, i.e., if either:
\begin{itemize}
    \item $(x-x_0)p(x)$ is not analytic at $x = x_0$, or
    \item $(x-x_0)^2q(x)$ is not analytic at $x = x_0$
\end{itemize}

\subsection{Key Differences}
\begin{enumerate}
    \item \textbf{Solution Method}:
        \begin{itemize}
            \item Ordinary points: Regular power series solutions exist
            \item Regular singular points: Frobenius series solutions exist
            \item Irregular singular points: Neither method works directly
        \end{itemize}
    \item \textbf{Behavior of Solutions}:
        \begin{itemize}
            \item Ordinary points: Solutions are well-behaved and analytic
            \item Singular points: Solutions may have branching or essential singularities
        \end{itemize}
    \item \textbf{Series Convergence}:
        \begin{itemize}
            \item Ordinary points: Power series converge in some neighborhood
            \item Singular points: More complex convergence behavior
        \end{itemize}
\end{enumerate}

\section{Difference using the power series and Frobenius method}
\subsection{Difference in starting index}
The difference in handling the starting index when using a normal power series and the Frobenius method is rooted in the nature of these two approaches and the types of points they are centered around.

\subsubsection{Normal Power Series}

\textbf{When to Use:}
\begin{itemize}
    \item The normal power series method is typically used when the differential equation has an ordinary point, where the functions involved are analytic and well-behaved.
\end{itemize}

\textbf{Series Expansion:}
\begin{itemize}
    \item Assume a solution of the form:
    

$$ y(x) = \sum_{n=0}^{\infty} a_n x^n $$


\end{itemize}

\textbf{Differentiation:}
\begin{itemize}
    \item When you differentiate this series:
    

$$ y'(x) = \sum_{n=1}^{\infty} n a_n x^{n-1} $$ 


    

$$ y''(x) = \sum_{n=2}^{\infty} n(n-1) a_n x^{n-2} $$ 


\end{itemize}

\textbf{Reason for Index Change:}
\begin{itemize}
    \item The starting index increases because differentiation reduces the power of $ x $ by one. As a result, the first few terms (often constants) disappear (e.g., the $ a_0 $ term vanishes in the first derivative).
\end{itemize}


\subsubsection{Frobenius Method}

\textbf{When to Use:}
\begin{itemize}
    \item The Frobenius method is used when the differential equation has a regular singular point, where the functions might not be entirely analytic, but the equation can still be solved using a modified series approach.
\end{itemize}

\textbf{Series Expansion:}
\begin{itemize}
    \item Assume a solution of the form:
    

\[
    y(x) = \sum_{n=0}^{\infty} a_n x^{n+s}
    \]


    Here \( s \) is a number determined by solving the indicial equation, which arises from substituting the series into the differential equation.
\end{itemize}

\textbf{Differentiation:}
\begin{itemize}
    \item When you differentiate this series:
    

\[
    y'(x) = \sum_{n=0}^{\infty} (n+s) a_n x^{n+s-1}
    \]


    

\[
    y''(x) = \sum_{n=0}^{\infty} (n+s)(n+s-1) a_n x^{n+s-2}
    \]


\end{itemize}

\textbf{Reason for No Index Change:}
\begin{itemize}
    \item The starting index doesn't change because the exponent \( n+s \) allows for non-integer powers and shifts in the series. The indicial equation ensures that the series accommodates the nature of the singularity, which maintains the index starting point.
\end{itemize}